

\begin{task}{1}[2/5]
    Орбитальная скорость Земли внезапно увеличилась в $\sqrt{2}$ раз и планета отправилась осваивать новое космическое пространство. Во сколько раз Земля будет быстрее Марса в момент пересечения его орбиты?
\end{task}


\image[c][0.5\textwidth]{problems/1/img/1.jpg}

\begin{solution}
    Так как скорость увеличилась в $\sqrt{2}$ раза, Земля вышла на параболическую орбиту, для скорости на которой для расстояния $r$ выполняется простое соотношение:
    $$
    v_{\Earth}(r) = \sqrt{\frac{2GM}{r}}.
    $$
    В то же время Марс двигается по круговой орбите, для которой скорость константная и равна
    $$
    v_{\mars}(r) = \sqrt{\frac{GM}{r}}.
    $$
    Тогда отношение скоростей Земли и Марса будет равно отношению скорости на круговой орбите к скорости на параболической орбите, то есть $\sqrt{2}$
\answer{$\sqrt{2}$} % ответ на задачу
\end{solution}
